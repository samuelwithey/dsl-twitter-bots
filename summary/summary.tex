\chapter{Summary and Reflections}

\section{Further Work}

Further extensions of the project would have been to include the functionality described in the desirable functional requirements. This includes implementing a front-end to the Django web application where users can interact with the domain specific language from a multi-user web page. The desirable functional requirements were not achieved due to time constraints and the complexity of the implementation of the system. \newline \par

Another extension of the project would be to include more automation and bot-scripts. The Tweepy API has powerful functionality which can be seen in the StreamListeners. StreamListeners obtains high volumes of tweets in real time and its functionality is demonstrated in FavRetweetListener in bot\_scripts. This functionality provides a lot of opportunities for automation and could be integrated with sentiment analysis in natural language processing to engage and interact with tweets based on its sentiment. There is vast potential for more complex and practical bot scripts to be implemented using the Tweepy API. \newline \par
 
The project meets Functional Requirement 1, as it provides all the functionality of what can be completed by a human on the standard web page. A limitation of this functional requirement is that the interaction of the basic functionality of Twitter is a complicated process. This is due to the Twitter API using Snowflake to generate unique IDs for objects within Twitter (Tweets, Direct Messages, Users). These ID's are often unknown and are not a viable or practical solution to interacting with the basic functionality of the API. Tweeting a user can manually be achieved by including their username following the standard \@ notation however to retweet, reply to tweet and favourite, the ID of the Tweet must be known. Further extensions of the project would look at solutions to solving this, including algorithms to search for ID's by usernames through the Tweepy API to create a more intuitive and natural interaction with the domain specific language.

\section{Conclusion}

The project was completed using an Agile SCRUM methodology. This methodology focuses on using short work cycles called sprints. Sprints throughout the project varied in length depending on what the sprint aimed to achieve in the sprint plan. The sprint plans used the GANT chart produced in the interim report to further break down tasks. This process of dividing the larger tasks into smaller sprint plans allowed for easy implementation of the project and was managed through a Trello board. This board included three columns, things to do, doing and completed. Each task within the column had a label and a sprint number to make it easy to identify which sprint cycle includes which tasks and what work is outstanding at the end of a sprint. This method of project planning was extremely effective, allowing each task to be easily managed and implemented and allowed for real-world delays and issues. The project timeline and management did not encounter many delays. The most significant delay for the management of the project was getting access to the Twitter API keys. It took several weeks to gain access to the keys and this delayed the implementation of the project as the application to gain access to the keys was not done until after the planning and designs. This delayed the implementation stage of the project as the API keys were vital for the implementation stage. \newline \par

The project was unable to meet the desirable requirements due to the underestimated amount of time it would take to successfully implement the domain specific language and the visitors. These limitations came from small amount of errors in the domain specific language which had to be refactored. Another factor was the choice to write the software in Python. ANTLR4 has Python 3 as a code generation target, but does not include any documentation for Python. This required all of the examples and documentation from The Definitive ANTLR 4 \cite{parr2013definitive} to be converted from Java to Python. This made implementation a slow process as the documentation and examples did not always directly translate to Python. In these scenarios it was required to view the ANTLR 4 Python source code to understand how to implement certain functionality and this was most prevalent when implementing the visitor functions. \newline \par

Other decisions to use certain tools and technologies added complications to the project. The decision to use the Python ANTLR generation was to use be able to easily include the domain specific language in the Django web application. As the front end of the Django web application was not implemented, it was not necessary to use Django. Django provides a simple way to manage the account credentials, execute the domain specific language, create Twitter Campaigns and execute tests. This is a nice feature and would work well with the future aims and objects and desirable requirements however for the project these features could easily be implemented without the use of Django. \newline \par

The original project ideas and designs had the intention of working across a wider range of social media platforms and not limited to Twitter. This project limitation came from the lack of access for the API and developer tools for other social media platforms. Other social media platforms such as Facebook and Instagram have limited access to their API and developer toolkits, where the specification of the project was not eligible for API access. \newline \par