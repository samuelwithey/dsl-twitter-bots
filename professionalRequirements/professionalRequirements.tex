\chapter{Professional Considerations}

The nature of the project using social media automation results in a large array of ethical and professional considerations. The aims and motivations for the project are for individuals and brands to allow more automation and freedom to manage and publish their social media content however the project could be used in a variety of negative ways. Examples of real world negative ethical consequences of social media automation are spam, harassment, spreading false information, troll armies and influencing content reach and impressions. \newline \par

Mitigating the risk of the tool being used for negative consequences is an intricate task to be implemented into the project. To access and interact with the Twitter API, users must have authorised consumer keys and access tokens by Twitter. This allows the project to be an extension of the Twitter API by only allowing users who have been pre-authorised to access the Twitter API. If the user of the tool violates the Developer Agreement and Policy \cite{DeveloperPolicy}, their access to the Twitter API would be revoked and the user would not be able to use the tool. \newline \par

Another ethical consequence of social media automation is spam. The Twitter API imposes rate limits, which is a per-user basis of how often users can invoke API methods. If these limits are exceeded, a time penalty will be added to the user and this is a way to combat spam. Since the implementation of the domain specific language does not contain any direct looping or recursion, it is not necessary to directly implement rate-limits into the standard functionality of the domain specific language. The rate limits have been imposed on the automated bot scripts to combat spam and to stay within the Twitter Developer Agreement and Policy \cite{DeveloperPolicy}. Tweepy, a Python library used for accessing the Twitter API in the project has its own functionality to restrict spam. When making posts, Tweepy requires each post to be unique. If a post is not unique, a duplicate error will be raised and the post will not be posted.

\section{BSC Code of Conduct}

Ethical standards governing the conduct of computing professionals in the UK are set out in the Code of Conduct published by BCS - The Chartered Institute for IT. The project adheres to all standards set by the code of conduct. Section 1.a, Public Interest, is an important standard as the use of automation could disregard the well-being of others and the project adheres to this standard through built in rate-limits. All of the source code for the project will be open source and publicly available on GitHub to uphold Section 4, Duty to Profession. This will be in direct accordance with section 4.f, to encourage and support fellow members in their professional development.