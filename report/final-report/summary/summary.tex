\chapter{Summary and Reflections}

The completed project met all of the mandatory functional and non-functional requirements stated in the requirements analysis. The domain specific language meets Functional Requirement 1, as it provides all the basic functionality of what actions can be completed by a human on the standard Twitter web page. The domain specific language meets Functional Requirement 3, 4 \& 5 through the use of the bot scripts. The project did not meet any of the desirable functional requirements. The functionality of the mandatory functional requirements is demonstrated in the evaluation section of the report and through Django's test suite. \newline \par

The original project ideas and designs had the intention of working across a wider range of social media platforms and not limited to Twitter. This project limitation came from the lack of access for the API and developer tools for other social media platforms. Other social media platforms such as Facebook and Instagram have limited access to their API and developer toolkits, where the specification of the project was not eligible for API access. \newline \par

Further extensions of the project would have been to include the functionality described in the desirable functional requirements. This includes implementing a front-end to the Django web application where users can interact with the domain specific language from a multi-user web page. The desirable functional requirements were not achieved due to time constraints and the complexity of the implementation of the system. \newline \par

Another extension of the project would be to include more automation and bot-scripts. The Tweepy API has powerful functionality which can be seen in the StreamListeners. StreamListeners obtains high volumes of tweets in real time and its functionality is demonstrated in FavRetweetListener in bot\_scripts. This functionality provides a lot of opportunities for automation and could be integrated with sentiment analysis in natural language processing to engage and interact with tweets based on its sentiment. There is vast potential for more complex and practical bot scripts to be implemented using the Tweepy API. \newline \par
 
The project meets Functional Requirement 1, as it provides all the functionality of what can be completed by a human on the standard web page. A limitation of this functional requirement is that the interaction of the basic functionality of Twitter is a complicated process. This is due to the Twitter API using Snowflake to generate unique IDs for objects within Twitter (Tweets, Direct Messages, Users). These ID's are often unknown and are not a viable or practical solution to interacting with the basic functionality of the API. Further extensions of the project would look at solutions to solving this, including algorithms to search for ID's by usernames through the Tweepy API to create a more intuitive and natural interaction with the domain specific language.