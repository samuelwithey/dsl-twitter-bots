\renewcommand{\chaptername}{}

\chapter{Introduction}

In January 2019, it was reported that there were 3.484 billion worldwide active users across all social media platforms. This rapid social paradigm shift has transformed consumer behaviour and social interaction through online social platforms which allow instantaneous direct communication. This has provided unbounded limitations for brands to connect and engage directly with their audiences.

\section{Motivation}

The motivation of the project is finding a unique solution around managing and automating social media content on Twitter. This is because of the limited functionality that the standard Twitter web page provides, with a lack of automation it makes managing large amounts of content and intricate tasks.

\section{Aims}

The aim of the project is to provide a solution to the limited functionality of Twitter by bridging the gap between the novice user and the Twitter API. This will be achieved by designing and implementing a domain specific language and an interpreter. This allows a lightweight language to provide the user with the core functionality of what can be achieved on Twitter with added functionality of the Twitter API including scheduling and automating content.

\section{Domain Specific Languages}

Domain specific languages are optimised for a given class of problems called a domain.  It is based on abstractions that are closely aligned with the domain for which the language is built and a domain specific languages syntax is suitable for expressing these abstractions concisely. This differs from general-purpose languages which are used by programmers to instruct computers. General-purpose languages are Turing complete, which means they can be used to implement anything that is computable by a Turing machine. \\

An advantage of using a domain specific language is that it is another layer of abstraction. Notation can be defined that expresses the abstractions concisely and makes interacting with programs easy and efficient. This is important for the project as the domain specific language can be used by non-programmers. It provides a clean level of abstraction that moves away from general-purpose languages and API’s which non-programmers are not competent enough to use and allows them to work with a language closely aligned with the domain they work in. This is a key solution to the problem area as it abstracts the complexity of using API’s by creating a language which is syntactically simpler than a general purpose language and can be used by non-programmers which is the target user for the system. \\

For the domain specific language to execute it requires an interpreter or compiler. When text is interpreted, it is parsed and the result from the program is produced in a single process. \\

A parser is the processing of structuring a text according to a given grammar. The parser will generate a syntax/parse tree which is a data structure which precisely shows how various segments of the program text are to be viewed in terms of the grammar. A grammar or context-free grammar are the formalism for describing the structure of programs in a programming language by describing the syntactic structure. Since the semantics of a language is defined in terms of the syntax, the context-free-grammar is also instrumental in the definition of the semantics. \\