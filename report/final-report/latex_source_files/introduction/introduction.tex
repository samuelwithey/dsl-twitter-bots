\renewcommand{\chaptername}{}

\chapter{Introduction}

\section{Problem Area}

In Q1 2020, there were 166 million Monetizable Daily Active Users (mDAU) worldwide on Twitter, a 5.2\% increase compared to Q4 2019 and a 24\% increase year-over-year \cite{twitter2020}. This rapid social paradigm shift has transformed consumer behaviour and social interaction through online social platforms which allow instantaneous direct communication. The social paradigm shift has provided unbounded limitations for brands and users to connect and engage directly with their audiences. The standard Twitter web application provides limited functionality for content management and automation, therefore managing large amounts of content on Twitter is an intricate task.

\section{Aims and Objectives}

The overall goal of this project is to provide a unique solution to the issues around managing social media content on Twitter. There are several tools which allow for social media content management; however, they are often closed source and hidden behind a pay-wall. Lack of access to these tools prevents the standard Twitter user from accessing the functionality of automation, making it difficult and a time-consuming task to manage large amounts of content. The project aims to design and implement an open-source domain specific language and an interpreter to configure and operate a variety of Twitter bots. This solution will bridge the gap between the novice Twitter user and the functionality of the Twitter API. The domain specific language will provide the user with the core functionality of the standard Twitter actions with the added functionality of the Twitter API, including scheduling and automating actions and content.

\section{User Needs}

The users of the system will see a novel solution to Twitter content management, automation and bots. Social media management tools and dashboards such as Hootsuite provide a single platform to schedule, monitor, analyse and curate social media content. The project aims to include the functionality that these social media management tools provide with the extension of automated bots. Twitter bots require knowledge of programming, web services, and the use of an API which is often outside the domain of the standard Twitter User. This solution allows non-programmers to interact with Twitter bots easily and utilise the Twitter API. The system will provide the functionality for users to upload a program with a syntax closely aligned to their domain, execute the program and have the tools to monitor the interactions of the bots and the social media content.

\section{Project Motivation}

The project motivation is to explore the concept of creating a small, lightweight language to automate Twitter content through the use of Twitter bots. The motivation to do this is to provide a unique solution to user needs through the use of industry-standard tools and services. The project aims to use the latest parser generator tools, web-frameworks, web-services and social media API's. The reason for using these tools and services is to develop my skills further and explore the concept of developing a solution based on fundamental principles of computer science and using modern technologies.

